\documentclass{beamer}

\usepackage[utf8]{inputenc}
\usepackage{csquotes}
\usepackage{latexsym,amsmath,xcolor,multicol,booktabs,calligra, animate, subfig}
\usepackage{graphicx,pstricks,listings,stackengine, bbm}
\usepackage{tikz}
\usetikzlibrary{fit,tikzmark}   
\usepackage{booktabs,cellspace}
\usepackage{color, colortbl}
\usepackage{hyperref}

%Information to be included in the title page:
\title{Analysis of Utah's Proposed Regulation}
\author{Lauren Beatty\\ lbeatty@edf.org}
\institute{Environmental Defense Fund}
\date{\today}
\logo{\includegraphics[height=1cm]{EDF_color_printCMYK_2022.png}\hspace*{.03\paperwidth}\vspace{.01\paperwidth}}


\begin{document}

\frame{\titlepage}

\begin{frame}
\frametitle{Does the Proposed Regulation Sufficiently Cover Plugging Liabilities?}
    Compare hypothetical bond amounts by firm against:
    \begin{itemize}
        \item estimated plugging costs for \textit{all} wells
        \item estimated plugging costs for inactive/marginal wells.
    \end{itemize}
    Summary of findings:
    \begin{itemize}
        \item If plugging costs are high, then the proposed regulation doesn't cover total liabilities for many firms.
        \item \textit{However}, the proposed regulation does appear to sufficiently cover plugging liabilities for marginal/inactive wells.
        \item Meeting higher tier requirements can provide significant cost savings for operators.
    \end{itemize}
\end{frame}

\begin{frame}{Methods/Assumptions}
\label{BondCalc}
To calculate hypothetical firm-level bonds I conduct the following steps:
\begin{enumerate}
    \item For each well in the state, sum production from 06/2022 until 06/2023 and classify each well as marginal ($<$2 BOE per day), active, or inactive.
    \item For each operator, calculate which tier they would fall in (determined by their percentage of inactive/marginal wells and total production for the year).
    \item Calculate their blanket bond and marginal/inactive well bond for each operator along with their per-well bonds for wells inactive $>12$ months.
    \item For operators that don't meet tier requirements, calculate their total bonds as the sum of well bonds using the April draft of the proposed regulation.
\end{enumerate}
\hyperlink{bondingnumbers}{\beamerskipbutton{Individual Bond Numbers}}
\end{frame}

\begin{frame}{Methods/Assumptions}
\label{costcalc}
    To calculate estimate plugging costs I make four cost assumptions:
    \begin{enumerate}
        \item Assume each well costs $\$37,500$ to plug.
        \item Assume each well costs $\$75,000$ to plug.
        \item Assume each well costs $\$6$ per well-foot of depth to plug.
        \item Assume each well costs $\$12$ per well-foot of depth to plug.
    \end{enumerate}
    \vspace{.7cm}
    The estimates produced by assuming $\$6$ per foot look largely like the estimates produced from assuming $\$37,500$ per well.  Similarly, the estimates assuming $\$12$ per foot look largely like the estimates from assuming $\$75,000$ per well.

    \hyperlink{depths}{\beamerskipbutton{Well Depths}}
\end{frame}

\begin{frame}{Frame Title}
    
\end{frame}

\begin{frame}{Thanks!}
    All of the code to reproduce these charts is available at: \href{https://github.com/lbeatty1/UtahDNRAnalytics}{https://github.com/lbeatty1/UtahDNRAnalytics}\\
    \vspace{1cm}

    Please feel free to contact me at:
    \href{lbeatty@edf.org}{lbeatty@edf.org}
\end{frame}

\begin{frame}{Individual Well Bond Numbers}
\label{bondingnumbers}
\begin{table}[]
\begin{tabular}{l|l}
Depth                                      & Bond    \\
\hline
\leq 1,000                          & 10,000  \\
\textgreater{}1,000 and \leq 3,000  & 20,000  \\
\textgreater{}3,000 and \leq 6,000  & 40,000  \\
\textgreater{}6,000 and \leq 9,000  & 65,000  \\
\textgreater{}9,000 and \leq 12,000 & 85,000  \\
\textgreater{}12,000                       & 110,000
\end{tabular}
\end{table}
\hyperlink{BondCalc}{\beamerskipbutton{Back}}
    
\end{frame}

\begin{frame}{Well Depths}
\label{depths}
\centering
    \includegraphics[scale=0.12]{Figures/DepthHistogram.jpg}\\
    \hyperlink{costcalc}{\beamerskipbutton{Back}}
\end{frame}
\end{document}